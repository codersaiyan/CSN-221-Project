\chapter{References}

\section{Register Renaming and Dynamic Speculation: an Alternative Approach}
by Mayan Moudgill \& Keshav Pingali Stamatis Vassiliadis Department of Computer Science, School of Electrical Engineering, IBM Corporation, Cornell University, Cornell University, Enterprise Systems, Ithaca, NY 14853. Ithaca, NY 14853. Poughkeepsie, NY 12602. September 15, 1993\\
\url{http://citeseerx.ist.psu.edu/viewdoc/download?doi=10.1.1.59.4027&rep=rep1&type=pdf}

In this paper, they implemented register renaming, dynamic speculation and precise interrupts. Renaming of registers is performed during the instruction fetch stage instead of the decode stage, and the mechanism is designed to operate in parallel with the tag match logic used by most cache designs. By folding renaming into the instruction fetch stage, instead of implementing it in the instruction decode stage as competing approaches do, they remove complexity from the already-complex decode stage. 

\section{An Efficient Algorithm for Exploiting Multiple Arithmetic Units}
R. M Tomasulo\\
\url{https://pdfs.semanticscholar.org/8299/94a1340e5ecdb7fb24dad2332ccf8de0bb8b.pdf}

\section{The Design Space of Register Renaming Techniques}
D.Sima\\
\url{https://ieeexplore.ieee.org/stamp/stamp.jsp?tp=&arnumber=877952&tag=1}