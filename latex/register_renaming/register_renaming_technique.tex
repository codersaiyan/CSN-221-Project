\chapter{Register Renaming Technique}
Register renaming is a technique used to allow multiple execution paths without conflicts between different execution units trying to use the same registers.Instead of just one set of registers being used,multiple sets are put into the processor.This allows different execution units to work simultaneously without unnecessary pipeline stalls. For many computations, a CPU uses the data stored in its registers.Processors with large no of registers can use them to hold many program variables,which reduces the number of cache and external memory accesses. Register renaming technique used in all current processors based on the x86 instruction set, can eliminate this false dependency by dynamically substituting different physical registers when dealing with closely-grouped independent operations. Internally, renaming requires more physical registers than the logical registers that are visible to the outside world.
Register renaming distinguishes two kinds of registers: logical and physical registers.Logical registers are those used by the compiler,whereas physical registers are those actually implemented in the machine. Typically, the number of physical registers is quite larger than the number of logical registers. When an instruction that produces a result is decoded, the renaming logic allocates a free physical register. The logical destination register is said to be mapped to that physical register. Subsequent data dependent instructions rename their source registers to access this physical register.


